\documentclass{article}

% if you need to pass options to natbib, use, e.g.:
%     \PassOptionsToPackage{numbers, compress}{natbib}
% before loading neurips_2018

% ready for submission
% \usepackage{neurips_2018}

% to compile a preprint version, e.g., for submission to arXiv, add add the
% [preprint] option:
%     \usepackage[preprint]{neurips_2018}

% to compile a camera-ready version, add the [final] option, e.g.:
     \usepackage[final]{neurips_2018}

% to avoid loading the natbib package, add option nonatbib:
%     \usepackage[nonatbib]{neurips_2018}

\usepackage[utf8]{inputenc} % allow utf-8 input
\usepackage[T1]{fontenc}    % use 8-bit T1 fonts
\usepackage{hyperref}       % hyperlinks
\usepackage{url}            % simple URL typesetting
\usepackage{booktabs}       % professional-quality tables
\usepackage{amsfonts}       % blackboard math symbols
\usepackage{nicefrac}       % compact symbols for 1/2, etc.
\usepackage{microtype}      % microtypography
\usepackage{amsmath}
\usepackage{tikz}
\usetikzlibrary{calc}
\usetikzlibrary{bayesnet}
\usetikzlibrary{arrows}
\usepackage{color}
\usepackage{array}
\newcolumntype{C}[1]{>{\centering\arraybackslash}p{#1}}
\newcolumntype{R}[1]{>{\raggedleft\arraybackslash}p{#1}}
\newcolumntype{L}[1]{>{\raggedright\arraybackslash}p{#1}}
\usepackage{caption}
\usepackage{subcaption}

\title{The Effects of Skew on Convolutional Bayesian Neural Network Performance}

% The \author macro works with any number of authors. There are two commands
% used to separate the names and addresses of multiple authors: \And and \AND.
%
% Using \And between authors leaves it to LaTeX to determine where to break the
% lines. Using \AND forces a line break at that point. So, if LaTeX puts 3 of 4
% authors names on the first line, and the last on the second line, try using
% \AND instead of \And before the third author name.

\author{%
  William Watson \\
  Johns Hopkins University\\
  \texttt{billwatson@jhu.edu} \\
}

\begin{document}
% \nipsfinalcopy is no longer used

\maketitle

\begin{abstract}
  We will examine the effect of a skewed classification task on Bayesian Neural
  Networks. We will attempt to perform parameter learning via Bayesian approaches
  to approximate parameter posterior distributions and compare the model's
  robustness to skew with a regular network.
\end{abstract}

\section{Introduction}
Consider the simplest neural network, consisting of a single feature $x \in \mathbb{R}$ and
prediction $y \in \mathbb{R}$ as follows:
\begin{equation}
  y = w \cdot x + b
\end{equation}
Usually, we would use frequentist approaches to solve for a point estimate of
the feature weight $w$ and bias term $b$, either through stochastic gradient
descent or ordinary least squares.
However, lets consider a Bayesian Neural Network with Gaussian priors:
\begin{equation}
  \centering
  \begin{split}
    w \; \sim & \; \mathcal{N}\left( \mu_w,\; \sigma^2_w \right) \\
    b \; \sim & \; \mathcal{N}\left( \mu_b,\; \sigma^2_b \right) \\
    y \; = & \; w \cdot x + b \\
    y \; \sim & \; \mathcal{N}\left( x \cdot \mu_w + \mu_b,\; x^2 \cdot \sigma^2_w + \sigma^2_b \right) \\
  \end{split}
\end{equation}
We can use a Bayesian approach to yield a posterior belief about our parameters
$w$ and $b$, which allows us to ascribe an uncertainty to our predictions with confidence intervals.
We can also sample models and aggregate predictions to form a more
robust predictor given the data.
However, the derivation of the posterior distribution can be intractable, and
hence we can use variational inference methods and Monte Carlo sampling as an
approximation to the true posterior.

\section{Data}
We plan to use the Rijkmuseum Dataset \cite{Rijksmuseum} to perform artist
classification. However, we will explore how skew will effect our model's
generalization to a test set. The Rijkmuseum Dataset is heavily skewed,
with 90 artists having over 200 pieces, Rembrant alone with 1,384 pieces, out of
a total collection with 6,629 artists. Hence this dataset will easily provide a
skewed setting for experimentation.

We will constrain the classification problem to fewer artists, filtering out
lesser-known artists and controlling the skew manually. We will train a model
using several methods, and perform an inference task on predicting the artists.

% More like inference
\section{Methods}
We will train our network using 2 Bayesian methods and compare it with a regular
neural network as our baseline.
\subsection{Stochastic Variational Inference (SVI)}
As detailed in Hoffman et. al \cite{SVI}, we can approximate the posterior
distributions by using \textit{Stochastic Variational Inference}. Here, the
task is transformed from a complex inference problem into a stochastic
optimization problem that follows noisy estimates of the gradient to
maximize the \textit{evidence lower bound} by repeatedly subsampling the data.
\subsection{Markov Chain Monte Carlo (MCMC)}
We will attempt a \textit{Markov Chain Monte Carlo} approach using the No-U Turn Sampler
(NUTS) \cite{NUTS} to provide a sampling-based method to extract the posterior
distributions of our parameters. We can compare both Bayesian approaches in
their choice of posterior distributions for our parameters.
\subsection{Maximum Likelihood Estimation (MLE)}
Our baseline model will consist of an equivalent CNN trained in the normal setting,
providing a point estimate through \textit{Maximum Likelihood Estimation}.
This will give us an idea on how skew effects point estimate learning dynamics.
% However, it is important to note that, from a Bayesian perspective, MLE is
% equivalent to \textit{Maximum a posteriori estimation} (MAP) with a uniform prior
% on the parameters.

\section{Software}
This project will make use of Pyro\footnote{http://pyro.ai}, a
probabilistic programming language using PyTorch\footnote{https://pytorch.org}
to enable a unified framework for building neural networks and Bayesian modeling.

% \section{Proposed Analysis}
% For this project, we expect to analyze several results:
% \subsection{Skew Performance}
% For each level of skew, we will evaluate the performance in terms of accuracy,
% and attempt to ascribe confidence intervals on our Bayesian models for confidence
% of prediction.
% \subsection{Parameter Estimates}
% We can analyze a subset of the parameters and graph the MLE point estimate alongside
% its posterior beliefs. These evaluation tests will allow us to determine if the
% model is confident and if our approximations can capture the true posterior
% effectively.

\section{Proposed Analysis}
For this project, we expect to analyze several results:
\subsection{Skew Performance}
For each level of skew, we will evaluate the performance in terms of accuracy,
and attempt to ascribe confidence intervals on our Bayesian models for predictions.
\subsection{Parameter Posterior Estimates}
Since we are working with CNNs, we can sample kernels that may provide interesting
insight into the distribution of feature extraction.

% ------------------------------

\bibliographystyle{abbrv}
\bibliography{proposal}

\end{document}
